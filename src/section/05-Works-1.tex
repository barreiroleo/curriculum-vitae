% LTeX: language=es-AR
% Work Experience
\section*{{\faSuitcase} EXPERIENCIA LABORAL}

\MySectionNoPic{2022}{Dev. Sistemas embebidos}{San Rafael}{Freelance - POSING}{
    Sistema ecualizador de iluminación para centro comercial de gran envergadura.
    Múltiples estaciones sensores de luz comunicados por Modbus con unidad central ecualizador.
    Planificación, diseño, construcción e integración con sistema maestro de control.
}

\vspace*{0.22cm}
\MySectionNoPic{2022}{Desarrollador Lua}{Open source}{Side project - Neovim}{
    Creador y mantenedor de ``LTeX extra'': Plugin para editor Neovim. Agrega soporte de ``Code
    Actions'' para el servidor de lenguaje LTeX. \url{github.com/barreiroleo/ltex_extra.nvim}.
}

\vspace*{0.22cm}
\MySectionNoPic{2021}{Dev. Web}{San Rafael}{Freelance - Vamos Mendocinos}{
    Diseño y desarrollo de Landing Page. Despliegue y configuración de servidor y dominio. Proyecto
    para ``Vamos Mendocinos''. \cvtag{Docker} \cvtag{Nginx} \cvtag{Figma} \cvtag{SaSS}
}

\vspace*{0.22cm}
\MySectionNoPic{2021}{Ing. Electromecánico}{San Rafael}{Pasante - Jet Oil Technology}{
    Pasantías realizadas en la empresa de servicios petroleros ``Jet Oil Technology''.
    Proyectos destacados: Cálculo y diseño mecánico de tanque de fluidos. Sistema de seguridad
    electrónico-mecánico para bombas electro sumergibles. Sistema de captura y registro datalogger
    para régimen de funcionamiento de motor Diesel.
}

\vspace*{0.22cm}
\MySectionNoPic{2018-2020}{Dev. Sistemas embebidos}{San Rafael}{Freelance - Dealer Ingeniería}{
    Desarrollo de sistemas embebidos para ``Dealer Ingeniería''. Proyectos destacados:
    Sistema embebido de tacómetro digital con señal de freno.
    Sistema embebido para detección de concentración de ácido sulfhídrico.
    Sistema embebido de detección de vibraciones con transmisión de señal por RF.
}

\vspace*{0.22cm}
\MySectionNoPic{2016 (sep)}{Dev. Sistemas embebidos}{San Rafael}{Freelance - Teatro Tajamar}{
    Desarrollo de sistemab embebido para marquesina de teatro. Incluye sistema de potencia
    multiplexado de alta frecuencia. Proyecto para teatro Tajamar, Academia de canto Valeria Lynch.
}

\vspace*{0.22cm}
\MySectionNoPic{2016 (ene - jul)}{Dev. Sistemas embebidos}{San Rafael}{Freelance}{
    Desarrollo de sistema embebido IoT para estación meteorológica. Involucra recolección de datos
    de diversos sensores en un radio de 30m. Múltiples clientes con conexión a servidor de bases de
    datos local. Base de datos MySQL sobre Linux. Desarrollado en conjunto con Ing. Luis Boccaccini.
}
